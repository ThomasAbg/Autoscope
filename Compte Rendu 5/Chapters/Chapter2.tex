\chapter{Accessibilité au projet sur internet}

Le projet étant libre, il est disponible sur Github sous licence GPL-2. Nous tachons d'accompagner nos différents dépôts d'une documentation claire permettant d'obtenir les informations suivantes~:
\begin{itemize}[label=$\bullet$]
	\item Une description brève et précise sur le contenu du dépôt et son rôle au sein du projet.
	\item Une explication de la procédure à suivre pour utiliser le contenu du dépôt.
	\item Une explication de la procédure à suivre pour travailler sur le dépôt.
	\end{itemize}

\section{Dépôt principal}

\url{https://github.com/thibaudledo/Autoscope}

\vspace{1cm}

Le dépôt est organisé comme suit~:
\begin{itemize}[label=$\bullet$]
	\item Branche {\href{https://github.com/thibaudledo/Autoscope/tree/master}{\codeinline{text}{master}}}~: Sources du logiciel principal et explications sur le projet dans son ensemble.
	\item Release {\href{https://github.com/thibaudledo/Autoscope/releases}{\codeinline{text}{alpha}}}~: Paquets et fichiers binaires pour utiliser le projet "out of the box" (release expérimentale).
	\item Branche {\href{https://github.com/thibaudledo/Autoscope/tree/hardware}{\codeinline{text}{hardware}}}~: Fichiers Blender de la structure du télescope et fichiers KiCad de la carte électronique du projet.
	\item Branche {\href{https://github.com/thibaudledo/Autoscope/tree/doc}{\codeinline{text}{doc}}}~: Documentations et datasheets des composants et éléments utilisés pour le projet.
	\item Branche {\href{https://github.com/thibaudledo/Autoscope/tree/latex}{\codeinline{text}{latex}}}~: Fichiers LaTex et \codeinline{text}{.pdf} des comptes rendus sur le projet.
	\item Branche {\href{https://github.com/thibaudledo/Autoscope/tree/hello_mod}{\codeinline{text}{hello_mod}}}~: Sources d'un driver helloworld servant d'exemple.
	\item Branche {\href{https://github.com/thibaudledo/Autoscope/tree/a4988_mod}{\codeinline{text}{a4988_mod}}}~: Sources du driver des contrôleurs moteur et des capteurs de fin de course des moteurs.
	\item Branche {\href{https://github.com/thibaudledo/Autoscope/tree/mpu_9250_mod}{\codeinline{text}{mpu_9250_mod}}}~: Sources du driver de la centrale inertielle.
	\item Branche {\href{https://github.com/thibaudledo/Autoscope/tree/mtk3339_mod}{\codeinline{text}{mtk3339_mod}}}~: Sources du driver du GPS.
	\end{itemize}

\newpage
\section{Dépôt du système d'exploitation de la Raspberry-Pi}

\url{https://github.com/thomaslepoix/meta-autoscope}

\vspace{1cm}

Il s'agit de la couche de métadonnées utilisées par Yocto pour construire le système d'exploitation Linux utilisé par la Raspberry-Pi du télescope. Une image pré-compilée du système d'exploitation figurera sur {\href{https://github.com/thibaudledo/Autoscope/releases}{la release \codeinline{text}{alpha} du dépôt principal}.

\vspace{1cm}

Le dépôt est organisé comme suit~:
\begin{itemize}[label=$\bullet$]
	\item Branche {\href{https://github.com/thomaslepoix/meta-autoscope/tree/rpi}{\codeinline{text}{rpi}}}~: Métadonnées Yocto et explications de comment compiler et installer le système d'exploitation.
	\item Branche {\href{https://github.com/thomaslepoix/meta-autoscope/tree/rpi-repo}{\codeinline{text}{rpi-repo}}}~: Données utilisées par Repo pour synchroniser le dépôt à d'autres dépôts de métadonnées Yocto utilisées pout construire l'OS.
	\end{itemize}

\section{Dépôt du plugin de Stellarium}

\url{https://github.com/thibaudledo/Autoscope-Stellarium-plugin}

\vspace{1cm}

Il s'agit du plugin de Stellarium contenant l'interface par laquelle l'utilisateur interagira avec le télescope. Un paquet pré-compilé pour linux de Stellarium incluant le plugin figure sur {\href{https://github.com/thibaudledo/Autoscope/releases}{la release \codeinline{text}{alpha} du dépôt principal}.

\vspace{1cm}

Le dépôt est organisé comme suit~:
\begin{itemize}[label=$\bullet$]
	\item Branche {\href{https://github.com/thibaudledo/Autoscope-Stellarium-plugin/tree/master}{\codeinline{text}{master}}}~: Sources du plugin, patch des sources de Stellarium et explications de comment compiler une version de Stellarium intégrant le plugin.
	\end{itemize}

